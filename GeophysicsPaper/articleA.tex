\title{The feasibility of using distributed acoustic sensors in surface seismic application}

\address{Colorado School of Mines, \\ 1500 Illinois Ave, \\ Golden, CO, 80401}

\author{Samir F. Jreij, Whitney J. Trainor-Guitton, and Jim Simmons}

\email{samirjreij@mymail.mines.edu}

\footer{Example}
% \lefthead{Jreij, Trainor-Guitton, \& Simmons}
\righthead{The feasibility of using distributed acoustic sensors in surface seismic application}

\maketitle

\begin{abstract}


  In this paper, an imaging technique that utilizes sparsely sampled, multi-component geophone data and a dense surface distributed acoustic sensor (DAS) acquisition is proposed. The PoroTomo survey at Brady's Natural Lab consisted of 238 multi-component geophones that are spaced anywhere from 60 meters to 150 meters apart. This proves to be a difficult migration problem with such sparse spacing. Both 2-D and 3-D numerical experiments are performed to test the feasibility of using multi-component geophone and DAS data together. In 2-D, a reflectivity model is created from the local fault model in the PoroTomo Survey. This provided a variety of structural dips to test the imaging technique. It was found that using an horizontal force rather than a vertical force with these models produced a much sharper resulting image. A quantitative analysis is further performed to provide an unbiased perspective on the results. The quantitative analysis utilized both energy norm image filtering and a convolutional neural network to prove that distributed sensors add value to imaging efforts with sparsely-sampled, multi-component geophones. The 2-D example is an idealized experiment. A more extreme example is performed in 3-D to confirm the conclusions made in 2-D. A methodology to model DAS data in 3-D is presented prior to showing examples of utilizing the two data types together for imaging. Quantitative analysis is also required for an unbiased perspective on  the results. The results from quantitative analysis show that utilizing DAS in surface surveys with a sparse multi-component geophone acquisition proves to be useful in reducing the number of false positives by a small fraction. A more regular experiment must be performed prior to making conclusions about the added value of DAS, so 2-D lines of fiber were utilized instead of the PoroTomo acquisition geometry. The 2-D DAS acquisition increases identifying the true positives significantly.


\end{abstract}

\section{Introduction}
Distributed acoustic sensing (DAS) is a technology that uses Rayleigh scattering in a fiber-optic cable to detect elastic signals when wave particle motion is parallel to the sensing fiber \citep{hornman2013field}. The two main components used in distributed sensing are the interrogator unit and the fiber-optic cable. The interrogator unit works as a light source and receiver. It sends a known pulse of light down the fiber. Nearly any type of existing fiber-optic cable can be used in conjunction with an interrogator unit. Small imperfections within the fiber cause backscattering of light. Strain events along the fiber cause this backscattering to change slightly when a wavefield approaches the fiber. The interrogator unit can measure the Rayleigh backscattering and relate it to the strain along the fiber.

\subsection{DAS Advantages}
DAS has many advantages in various industries. For one, DAS is a low-cost acquisition system in wells that already contain fiber optic cables. Even in those wells that do not already contain fiber optic cables, a DAS vertical seismic profile (VSP) is often more affordable than renting and deploying geophones \citep{mateeva2014distributed}. DAS also enables seismic surveys to be acquired with dense sampling (as small as 10-centimeter receiver spacing) at large distances (tens of kilometers long). Achieving even 1-meter sampling with conventional geophone is expensive and logistically difficult. Lastly, DAS has almost perfect repeatibility in 4-D surveys when cemented in a borehole or trenched in the subsurface \citep{mateeva2013distributed}.

\subsection{DAS Disadvantages}
Although DAS may seem like the solution to seismic acquisition, it also has many disadvantages. DAS is most sensitive to waves that have particle motion parallel to the orientation of the fiber, so it is said that the technology has broadside insensitivity.  Multi-component geophones also have this issue; with more recording components, however, they are able to resolve more of the wavefield and are not affected by this broadside insensitivity as much.

Another disadvantage is that DAS coupling is not trivial in all environments. In a borehole environment, DAS can be cemented behind casing or permanently installed on production tubing \citep{mateeva2013distributed}. Surface distributed sensor coupling is a more challenging issue. \citet{lindsey2017fiber} describe how fibers can be utilized in loosely coupled environments. The Stanford Fiber Optic Array consists of a 2.5 km long array that lies in a conduit about 1 to 2 meters below ground. The DAS fiber geometry is restricted by the conduits, though, and the task of installing the fiber is more difficult if there are no existing conduits. \citet{daley2013field} have trenched the cable and returned at a later time to shoot the seismic survey. Although this method is effective, waiting to shoot a survey at a later time can be inconvenient.

\subsection{Literature Review}
Historically, DAS has been used in a borehole environment for flow monitoring, temperature measurements, and vertical seismic profiles \citep{clarke1983fiber, krohn2000fiber,mestayer2011field, barberan2012multi, cox2012distributed, daley2013field, mateeva2014distributed}.

\citet{mestayer2011field} describe how permanently installed fiber-optic infrastructure in existing wells can enable low-cost non-intrusive geophysical monitoring. Geophones generally only acquire data along a short subset of the well. This makes repeatable time-lapse surveys difficult as placing the geophones in the same location is not trivial. \citet{mestayer2011field} also discuss how borehole DAS is able to improve repeatibility and time-lapse sensitivity because it is able to acquire data along the full well with a single shot.  \citet{mateeva2014distributed} also describe a time lapse, 3-D DAS VSP application. They conclude that DAS has many major business impacts on fields that require enhanced oil recovery (EOR) including cost efficiency, safety, and synergy with other fiber optic applications.

\citet{barberan2012multi} discuss different ways DAS fiber can be coupled in a borehole environment. DAS fiber can be clamped to production tubing and used as a downhole seismic sensor. \citet{barberan2012multi} expand on this explaining that acquiring seismic data over the entire well is essental for acquiring additional transit times for velocity inversion and it allows for a wide range of incidence angles in terms of wave directions that arrive at the fiber for inversion. \citet{cox2012distributed} also looks at borehole DAS data, but now for the purpose of imaging. \citet{cox2012distributed} examine images produced from both geophone and DAS data types in a borehole example. They conclude that images produced from distributed sensing are similar to those produced by geophones and that DAS has potential to replace geophones.

\citet{daley2013field} describes field tests from both horizontal and borehole distributed sensors. They conclude that the SNR in surface DAS is not sufficient for observing P-waves and that DAS is more useful in borehole environments or longer surface arrays. They don't go into detail, however, regarding why certain waves are not observable in DAS.

As seen in these examples, DAS research has emphasized acquiring data in borehole environments because many wells are already equipped with fiber for production. As a result, acquiring DAS in boreholes is as simple as connecting the existing fiber-optic cable to a new interrogator unit that senses acoustic signal. Although there are some studies on surface DAS acquisitions \citep{daley2013field, hornman2017field}, there has not been a thorough study in active source experiments.

Most recent papers in DAS fiber focus on how the system can be used independently of geophones. This is viable in a borehole environment due to the fact that the predominant particle motion of the recorded seismic data is in the vertical direction, parallel to the fiber. Surface horizontal DAS is sensitive to the horizontal component of particle motion. P-wave reflections will not be recorded on surface DAS at normal incidence, assuming a flat-layered earth, since the particle motion is vertical. \citet{daley2013field} experiment with a vibroseis injecting a vertical force source. The reflected P-wave is not recorded on the DAS fiber as the experiment only had 1,000 meters of offset, and, therefore, the authors concluded that the SNR in surface DAS is insufficient for observing P-waves due to the relatively small incidental reflected angle. Other source mechanisms must be investigated before such a conclusion can be made about the feasibility of using surface DAS fiber. Another option is utilizing the DAS fiber with geophones to attempt to minimize the insensitivity of some waves. This paper explores different experiments using the field geometry from the PoroTomo survey in Northwest Nevada and numerical modeling.

\subsection{PoroTomo Survey}
The PoroTomo survey involved four-weeks of data acquisition of geodesy, interferometric synthetic aperture radar (InSAR), hydrology, temperature sensing, passive source seismology, and active source seismology data. These data were jointly collected to characterize and monitor changes in the rock mechanical properties of Brady's Natural Laboratory (BNL), an Enhanced Geothermal System (EGS) reservoir.

This paper investigates the active seismic source component of the PoroTomo Experiment. The PoroTomo survey is one of the most unique seismic acquisitions for surface DAS fiber. The survey included 238 multi-component geophones, 156 three-component (vertical and orthogonal horizontal) vibroseis source locations that swept from 5 to 80 Hz in 20 seconds, 300 meters of borehole DAS, and nearly nine kilometers of surface fiber-optic cable. The survey geometry is shown in Figure~\ref{fig:dasngeophone}.

\plot{dasngeophone}{width=\textwidth}{PoroTomo survey geometry. Green dots represent source locations, red dots represent geophone locations, and the blue line represents the surface DAS layout.}

The DAS fiber was placed in a 1-meter deep trench in January of 2016 to settle for the March seismic data acquisition. This is one of the first experiments to allow two months for the fiber to settle. This is also one of the first experiments to investigate the surface DAS sensitivity in 3-D and 4-D. BNL has been maintained as a geothermal resource and operated as a power plant by Ormat Technologies for approximately 25 years. During the four-week PoroTomo survey, Ormat operated the geothermal system at different pumping and producing rates in the injection and production wells allowing for a small-scale, time-lapse survey.

\subsection{Outline}
This paper explains, and quantitatively analyzes, different approaches for using surface DAS in active source seismic acquisitions. First, the theory behind DAS fiber and  wave types that DAS fiber can record is introduced using theoretical examples. Then, 2-D numerical modeling is utilized to perform a feasibility study on combining DAS and multi-component geophones data to create one image. This is theoretically useful as the multi-component geophones sample the full wavefield (Z, X, and Y components of particle motion) at sparse locations and the DAS fiber samples one component of the wavefield densely. The 2-D examples shown are idealized experiments. More extreme examples must be performed in 3-D prior to making any conclusions with the hypothesis previously presented, so the challenges with modeling DAS are discussed and a numerical modeling example is presented to test if adding DAS data can improve a multi-component DAS image in 3-D.

\section{Fiber Sensitivity}
Understanding how DAS fiber works is essential to working with the data that is currently available and for future surveys. For a conventional DAS seismic survey, a known pulse of light is sent into the fiber using an interrogator unit and some of the light is naturally scattered back due to imperfections within the fiber. The interrogator unit is able to record this scattered light along the fiber up to 10-kilometers away. This is known as the base condition inside of the fiber. The fiber undergoes a strain when a seismic wavefield approaches and a scattering of light is produced that is different from the base condition. The interrogator unit is able to relate this new scattering of light to local strain along the fiber by recording the time of arrival and the phase-lag of the returning light signals \citep{Parker2014}.

As stated previously, DAS fiber is most sensitive to waves that are able to stretch and squeeze the fiber, so the waves have to have particle motion parallel to the orientation of the fiber. Every seismic sensor has its own distinct sensitivity to the various types of waves depending on their emergent angle. The emergent angle ($\theta$) represents the angle between the incoming wave and the surface of the Earth. Consider a plane wave reflection in the X-Z plane: an emergent angle of $0^{\circ}$ represents a wave arriving parallel to the surface (or a plane wave traveling in the Z-direction); an emergent angle of $90^{\circ}$ represents a wave arriving perpendicular to the surface (or a plane wave traveling in the X-direction).

These points can be demonstrated with a simple 2-D example. Consider wave propagation in the x-z plane in a homogeneous, flat-layered, isotropic or vertical transverse isotropic medium (Figure~\ref{fig:Preflection2}-\ref{fig:SVReflection2}). The horizontal DAS fiber is oriented in the x-direction. P-waves have particle motion parallel to the direction of wave propagation \citep{aki1980quantative}. Normal-incidence reflections from a horizontal reflector will arrive perpendicular to the surface fiber.  In the case of a 2-D line of horizontal fiber with an vertical vibe, the reflected P-wave will not be seen at short offsets (Figure~\ref{fig:Preflection2}). The particle motion of P-waves is parallel to the direction of propagation, so at short offsets, the reflected P-wave will arrive perpendicular to the fiber. As seen in Figure~\ref{fig:TotPwave}, P-waves with a $0^{\circ}$ emergent angle show 0 amplitude on fiber and maximum amplitude on the z-component of a geophone. Moving to further offsets yields emergent angles that are at a larger angle to the fiber. According to Figure~\ref{fig:TotPwave}, these waves will show more data than waves arrive perpendicular to the fiber as they are propagating in the direction of the fiber and less data on the vertical component of the geophone. The further the offset, however, the lower the amplitude of the wave due to attenuation effects.

Shear-waves are potentially more interesting when recording with horizontal fiber. Consider again 2-D wave propagation in the x-z plane in a homogeneous, flat-layered, isotropic or vertical transverse isotropy medium (Figure~\ref{fig:SVReflection2}). SV-waves have particle motion in the x-z plane, as do P-waves \citep{aki1980quantative}. Normal-incidence reflections from a horizontal reflector will arrive perpendicular to the surface fiber. P-wave particle motion, as stated previously, will be in the z direction, and consequently, will not be recorded by the fiber. SV wave particle motion will be in the x-direction (emergent angle of $0^{\circ}$), and the DAS response will be maximum (Figure~\ref{fig:TotSwave}). At larger offsets, the SV-wave emerging angle begins to approach $90^{\circ}$. A larger emerging angle means less signal (Figure~\ref{fig:TotSwave}) is recorded by both the surface DAS and the x-component of the geophone because the SV-wave particle motion approaches perpendicular to these components (Figure~\ref{fig:SVReflection2}).

\multiplot{2}{Preflection2,SVReflection2}{width=\textwidth}{Consider a source that generates both P and S waves; this is a 2-D Homogeneous, flat-layered, isotropic ray  path example. The blue lines represent ray paths of the labeled waves, the blue arrow represents the propagation direction of the wave, the green line represents a horizontal reflector in the subsurface, and the yellow star represents the source. (a) Demonstration of P-P wave effect on fiber using ray paths. Particle motion is inline with propagation direction (blue arrow). The fiber will only record data at large offsets. (b) Demonstration of P-SV wave effect on fiber using ray paths. Particle motion is perpendicular with the propagation direction (green arrows). The fiber will only record data at short offsets.}

\multiplot{2}{TotPwave,TotSwave}{width=\textwidth}{Amplitude versus emerging angle for both geophone (blue) and horizontal DAS (red). An emerging angle of $0^{\circ}$ indicates a wave that is propagating perpendicular to the surface and an emerging angle of $90^{\circ}$ degrees indicated a wave is propagating parallel to the surface. (a) Sensitivity with respect to a P-wave at different emerging angles of horizontal DAS (red) and geophone z-component (blue). (b) Sensitivity with respect to an SV-wave at different emerging angles of horizontal DAS (red) and geophone x-component (blue).}

We also consider using SH-waves with the same 2-D survey geometry (homogeneous, flat-layered, isotropic). SH-waves propagate in the x-z plane. SH waves have particle motion perpendicular to the direction of wave propagation or, in this case, in the y-direction. The SH-DAS response will be zero since the SH particle motion is perpendicular to the DAS fiber in the y-direction. In this 2-D case, the SH-wave will be out of plane regardless of source-receiver offset. In 3-D, SH-waves can be seen on the DAS if they are properly oriented. For example, a source-receiver azimuth perpendicular to the 2-D fiber (in this case, in the y-direction) will produce a maximum amplitude reflection on the DAS since the particle motion is in the x-direction for all offsets. As the source-receiver azimuth moves inline with the fiber, the SH-wave particle motion decreases, and is equal to zero when the source-receiver azimuth is inline with the fiber.

% In this section, DAS fiber directionality was described analytically and depicted graphically. The directionality is important to understand when creating a seismic survey geometry.

\section{2-D Numerical Modeling Examples}
Imaging the geophone data is a difficult task in the PoroTomo Survey due to the irregular spatial sampling and offset. This paper focuses on identifying a way to resolve the spatial sampling issue. Fortunately, the PoroTomo survey includes surface DAS cable that has 10-meter gauge-length and an equivalent of 1-meter receiver spacing along the fiber. Many papers in the literature are interested in methods to convert DAS measurements (strain or strain rate) to a geophone equivalent (particle velocity or displacement) with the intent to replace point sensors with distributed sensors, or use existing geophone processing to clean up DAS data \citep{daley2013field,daley2015field,jreij2017field}. The idea of using both data types in simultaneous imaging is explored in this paper to produce more detailed images using synthetic examples.

\subsection{2-D Synthetic Design}
\citet{siler2013three} mapped the faults of Brady's Natural Lab shown in Figure~\ref{fig:FaultModelCornerView}. It is important to image these faults in detail as they are driving factors behind the recharge of the geothermal reservoir \citep{feigl2017overview,folsomimaging}. For this reason, a slice is taken from one of the wells Brady's Natural Lab \citep{siler2013three} in the PoroTomo Survey are used as reflection velocity models. This slice is shown in Figure~\ref{fig:dtestVP}. The \citet{siler2013three} fault model is used as a reflectivity model as it contains a variety of structural dips. It is important to test the structural dip imaging extremes with new methods to see how they will work in complicated subsurfaces.

\plot{FaultModelCornerView}{width=\textwidth}{\citet{siler2013three} fault density mapping within the PoroTomo Survey box. This model was used as a reflectivity model for the experiments within this section.}

\plot{dtestVP}{width=\textwidth}{Reflectivity model from \citet{siler2013three} extracted at the location of Well 56-A1 used for simulating data. Blue dots represent source locations and the red dots represent geophone locations. DAS fiber was placed between the geophone locations.}

Seismic sources in the PoroTomo experiment are not on a uniform grid. In fact, the source spacing is as large as 150 meters. The aim of this paper is to discuss how DAS and multi-component geophones can be used together to create a more detailed image. Seismic illumination describes how much of the subsurface can be imaged given a source-receiver geometry and velocity model. Illumination in seismic surveys is highly influenced by source-receiver spacing. For the purpose of this section, a constant source spacing of 75 meters (which is about the average source spacing in the PoroTomo survey) is used to minimize migration artifact effects from poor illumination. A 20 Hz Ricker wavelet is utilized at every source location. As discussed in the previous section, certain wave reflections are better for surface DAS experiments. Different sources will cause different reflection event amplitudes and directions. For the 2-D experiments present in this paper, both vertical and horizontal force sources are modeled.

2-D elastic forward modeling is used to produce strain and displacement data along the surface of our 2-D example excited by a vertical force source. Receivers at every one meter across the experiment are used for recording. This represents a 1,500 meter long, 2-D surface DAS line. As seen in Figure~\ref{fig:dasngeophone}, the PoroTomo survey did not include a straight fiber that was this long. Instead, the fiber was placed in an attempt to capture a variety of reflection data azimuths. It did include, however, a maximum offset of 1,500-meters across the entire survey. For this reason, this whole offset is included for the 2-D example.

The code generated for these experiments outputs both strain and displacement at every receiver location. Receivers are recorded orginally at every one meter across the experiment. The geophones are not sampled every one meter in the PoroTomo survey. The average geophone spacing of about 70 meters. A geophone spacing of 100 meters is chosen to analyze geophone spacing closer to the extremes of this experiment. The data are generated from a reflectivity model that is derived from Brady's fault model using an elastic FDM operator from the Madagascar package \citep{fomel2013madagascar}. The next step is to back propagate the recorded data from this forward modeling to recover the receiver wavefield. If this was a field experiment, the field data would be back propagated. The adjoint elastic operator is utilized rather than the forward elastic operator to obtain a more exact solution to the imaging problem. Two different sources are needed to create the receiver wavefield. An acceleration force is used for back propagation of the geophone data and a stress tensor is used for back propagation of the DAS data. The proper way to do imaging is to back propagate the data simultaneously, but this was not possible with current codes, so the data are back propagated individually.

The last wavefield that needs to be generated is the source wavefield. The source wavefield is a forward model from the original source location through a smooth velocity model. It is important that the velocity model is smooth as reflections will cause an improper final image. Now, a source and two receiver wavefields exist. An imaging condition is required to combine the wavefields.

Traditionally, the zero-lag, cross correlation imaging condition (IC) is used to create a migrated image \citep{claerbout1985imaging}. Although this methodology may provide a solution for elastic imaging, this IC produces four resulting images (PP, PS, SP, SS). This proves to be a more difficult comparison between different data types for the purpose of this paper. \citet{rocha2016isotropic} describes the use of an energy-norm based IC that exploits wavefield directionality to create one final elastic image that represents the measure of reflected energy. There are many other benefits to using the energy-norm IC, but the key is that one final image allows for an easy comparison of migrated elastic data.

The image produced from the elastic energy norm RTM with sparsely sampled multi-component geophones is shown in Figure~\ref{fig:dtestEIMGS}. This image shows reflectors are discontinuous and difficult to follow. The image is also covered with migration artifacts due to insufficient sampling of the wavefield. An example of this is presented around 800 meters on the x-axis of Figure~\ref{fig:dtestEIMGS}: the migration artifacts make it difficult for an interpreter to follow the shallow reflector. The deeper reflector in Figure~\ref{fig:dtestEIMGS} is impossible to identify.

The image produced from the elastic energy norm RTM with DAS fiber along the surface of the model creating a virtual receiver at every one meter is shown in Figure~\ref{fig:dtestEIMGSDAS}. The shallow reflector in this image is sharp and continuous, allowing for easy interpretation. Although migration artifacts are still present around 800 meters on the x-axis, these are different from those experienced in Figure~\ref{fig:dtestEIMGS}. These migration artifacts are now due to fake modes present because the wavefield is extrapolated using only the x-component data that was recorded with DAS fiber.

Now there are two images with two different migration artifacts (i.e. types of noise). The power of stacking the images should theoretically reduce the noise and highlight the reflection events. Linearly stacking the events, however, will not currently work as the amplitudes are on different scales. Instead, the amplitudes of both images are normalized by the maximum and then stacked to produce Figure~\ref{fig:dtestTot}. Although Figure~\ref{fig:dtestTot} still has artifacts in it, the reflectors are enhanced and the image is easier to interpret than Figure~\ref{fig:dtestEIMGS} or Figure~\ref{fig:dtestEIMGSDAS}.

\multiplot{3}{dtestEIMGS,dtestEIMGSDAS,dtestTot}{width=.47\textwidth}{(a) Resulting image from migrating geophone synthetic data. (b) Resulting image from migrating DAS synthetic data. (c) Combined image from migrating DAS and geophone synthetic data.}

\subsection{Value of Information}
All of the experiments presented in the paper can be qualitatively analyzed and discussed, but qualitative analysis is always different between people due to different biases and perspectives. A method to quantitatively analyze the experiments is needed to do effective comparisons.

The Value of Information (VOI) is a quantitative tool that originates from the field of decision analysis to quantify how relevant and reliable an information source is \citep{trainor2013value}. VOI estimates the possible increase in expected utility by gathering information. It is calculated by subtracting the prior value ($V_{prior}$) from the value with imperfect information ($V_{imperfect}$) shown in  Equation~\ref{eqn:VOI}.

\begin{equation}
  VOI=V_{imperfect}-V_{prior}
\label{eqn:VOI}
\end{equation}

The goal of this project is to observe if there is any added value to using distributed acoustic sensing in surface acquisitions. The value with imperfect information shown in Equation~\ref{eqn:Vimperfect} can only be calculated with a quantitative measure of how accurate the information source is.

\begin{equation}
V_{imperfect} = \sum_{j=F,NF} Pr(\theta^{int}=\theta_j)  {\max_a [\sum_{i=F,NF} Pr(\theta=\theta_i | \theta^{int}=\theta_j)v_a(\theta_i)]}
\label{eqn:Vimperfect}
\end{equation}

This quantitative measure can be represented by the posterior probability, $Pr(\theta=\theta_i | \theta^{int}=\theta_j)$, of the value with imperfect information equation \ref{eqn:Vimperfect}. Specifically for these problems, the posterior probability can be how often interpretations of faults align with the actual presence of faults. It is important to calculate the posterior reliability so the value of imperfect information can be completed. The posterior probability can be calculated using Equation~\ref{eqn:Posterior},

\begin{equation}
  Pr(\theta=\theta_i | \theta^{int}=\theta_j)=\frac{(Pr(\theta=\theta_i)) Pr(\theta^{int}=\theta_j | \theta=\theta_i)}{Pr(\theta^{int}=\theta_i)}; \forall i,j={F, NF}
\label{eqn:Posterior}
\end{equation}

\noindent
where $\theta$ represents a true value of Fault or Not Fault, $\theta^{int}$ represents an interpreted Fault or Not Fault. There are a variety of methodologies to produce information about whether an interpreted fault is actually a fault or not. The two that are utilized in this paper are energy filtering and machine learning.

\subsubsection{Energy Filtering}
As discussed previously in this paper, the resulting image from energy norm reverse time migration represents the relative amount of energy reflected in the subsurface. All images are simply a matrix of these relative reflected energy values. In theory, the largest amplitudes from this image will be reflection events. In the case of this experiment, the reflection events represent the fault plane targets of imaging. An amplitude filter is used as a first pass to interpret reflections in the dataset to calculate the posterior distribution.

Every model cell that is above the applied limit is assigned a value of 1 and every model box that is below the limit is assigned a value of 0. For example, the image in Figure~\ref{fig:dtestEIMGS} shows that the maximum absolute amplitude is about 500,000. The top 80\%, 90\% and 95\% of the reflected energy are filtered on the image to highlight areas where reflections are coming from as opposed to migration artifacts. The results are shown in  Figure~\ref{fig:EnergyMedCut} for the top 90\% of the reflected energy. Although the results in Figure~\ref{fig:EnergyMedCut} are not perfect representations of where interpreters would place the faults, this approach represents the beginning steps to quantify the value added by DAS fiber with sparse multi-component geophones.

\plot{EnergyMedCut}{width=\textwidth}{Reflection amplitudes that represent the top 90\% of energy recorded.}

A cell-by-cell comparison between the filtered energy images (Figure~\ref{fig:EnergyMedCut}) and the original fault image (Figure~\ref{fig:dtestVP}) is performed to identify how accurate both technologies are able to identify features in the fault model. The results of this cell-by-cell comparison are presented in confusion matrix form (Table~\ref{table:CONF}). A confusion matrix is a table that describes performance on a set of test data for which the true value is known. The term $\theta_F$ denotes that an actual fault exists and the term $\theta_{NF}$ denotes that no fault exists. The terms $\theta_F^{int}$ and $\theta_{NF}^{int}$ represent the interpretations of fault and no fault, respectively, based on energy filtering of images. The columns of the confusion matrix represent predicted classifications ($\theta_F^{int}$ and $\theta_{NF}^{int}$). The rows of the confusion matrix represent actual true statement of the subsurface ($\theta_F$ and $\theta_{NF}$).


\begin{table}[]
\centering
\caption{Confusion matrix for top 90\% energy reflected.}
\label{table:CONF}
\setlength{\tabcolsep}{1em}
\begin{tabular}{|l|l|l|}

   \multicolumn{3}{c}{Top 90\% energy reflected}\\
  \hline
            & $\theta_F^{int}$ & $\theta_{NF}^{int}$ \\
            \hline
 $\theta_F$ & 307 & 336 \\
\hline
 $\theta_{NF}$ & 922 & 21685 \\
\hline
\end{tabular}
\end{table}

The confusion matrices assist in calculating the posterior value using Equation~\ref{eqn:Posterior}. The posterior value gives the probability that an event which the data type predicted is the event present. The posterior can then be used to calculate the utility or value of information added when using DAS and geophone versus only geophone with Equation~\ref{eqn:VOI}. The results for the posterior values are displayed graphically in Figure~\ref{fig:POSTERIORS-vsource} for the vertical source. In every experiment, adding distributed sensors increases the probability of finding if a cell is a fault or not a fault (true positive or a true negative) and decreases the probability of a cell being a false positive or false negative.

Energy norm imaging allowed for an automatic method to interpret images output from the migration images. Filtering images based on amplitudes, however, is a crude approximation of how an interpreter would ``interpret" an image. A more sophisticated method to interpret images is required to get a better quantitative analysis of the imaging technique, so a machine learning classification scheme is examined.

\subsubsection{Convolutional Neural Network Analysis}
Machine learning is a field within computer science that focuses on the ability of computer systems to learn patterns within data without being explicity programmed for these patterns \citep{samuel1959some}. Machine learning has had a large boom in the geophysics industry within the last 10 years. The reasons for this are quite apparent: geophysicist work with large amounts of data, the geophysics field needs more quantitative analysis rather than qualitative, and machines are much better at identifying weak or high-dimensional patterns than humans are.

There are a variety of machine learning algorithms that can be utilized based on the problem that needs to be solved. One of the most powerful machine learning algorithms is the neural network. Neural networks are inspired by the biological neural networks that constitute human brains or at least how humans perceive they work \citep{Gerven2018}. They consist of many layers in parallel and every layer consists of a number of nodes. All neural networks consists of at least two layers: the input layer and output layer. All the extra layers in between the input and output layers are the hidden layers. The nodes of every layer are like neurons in the brain. Every neuron has its own activation function that determines whether it should be ``fired" or not similar to how a neuron in the brain behaves. Each layer receives the output from the previous layer based on if the previous neuron is fired or not.

Convolutional Neural Networks (CNN) in particular are at the core of most state-of-the-art computer vision solutions for a wide variety of tasks. One of the most accurate CNN image classifiers is the Inception-v3 model \citep{Szegedy2015}. The original Inception-v3 model is trained and tested on the 2012 ImageNet Large Scale Visual Recognition Challenge \citep{ILSVRC15}. The training dataset consisted of 10,000,000 labeled images that depicted 1,000 object categories. The Inception-v3 model was able to perform with 3.5\% top-5 error, meaning that the target label is within the top-5 probability classifications that the algorithm produced. A top-5 error of 3.5\% means the Inception-v3 model is able to perform with high accuracy, making it a top contender for a geophysics image classification problem.

The Inception-v3 model utilizes transfer learning which means it stores knowledge gained from training on the ImageNet dataset and then applies it to a different but related problem. It is difficult to train a CNN from scratch because a large dataset is needed with a substantial amount of machines equipped with GPU's. Instead, the intermediate layers of the Inception-v3 model are used as they are already trained on detecting edges, shapes, and other high level features. The last layer of the model is retrained to identify if an image is either a fault or not a fault.

Figure~\ref{fig:Inceptionv3A} shows the architecture of one module in the Inception v3 model. The full architecture has a large amount of modules similar to the one shown in Figure~\ref{fig:Inceptionv3A} that decompose the image into smaller subsets and apply different filters. Some of these subsets can be as small as one pixel by one pixel, while others are as large as the image size.

\plot{Inceptionv3A}{width=\textwidth}{Inception v3 model architecture for one module modified from \citet{Szegedy2015}. This specific module performs filters on three pixel by three pixel as well as one pixel by one pixel subsets of the original image. In parallel to these filters is a pooling layer that performs an operation such as an average of all cells in subsets of the image.}

Within each image, there are a variety of features that contribute to the final classification. It is not feasible for the algorithm to identify these features using pixel by pixel methodologies. Instead, pooling allows for a smaller subset of the image to be analyzed for a certain statistic such as mean, maximum, or minimum. Strides in the convolutional neural network world are how much the filter that the network is applying is shifted. After learning a certain portion of the image, the network must perform a stride to get to the next location. These strides are important when analyzing a variety of images as they dictate the search area. Convolutional layers perform filtering operations on the image or combine two images that have been filtered previously. One way to apply convolution is in the sense of an edge detector: the original image is convolved with a kernel matrix that represents an edge filter. Inception-v3 utilizes convolutional layers like the edge filter to identify patterns within images.

The Inception v3 model's ability to identify features can be leveraged within the geophysics realm. The first step is to create some training data to retrain the model. The objective is to see if DAS helped identify more faults than a sparse array of multi-component geophones. For the experiments in this chapter, RTM images are created from 2-D reflectivity slices of the \citet{siler2013three} fault model. There are about 500 other slices along both the X and Y axis of the PoroTomo grid. A number of these slices can be migrated to create training data for identifying faults.

The next step is to take windows of the migrated images and label them based on if there are faults or not within the image. 100 meter by 100 meter (10 grid cell by 10 grid cell) subsets of the migrated images were created. There are a large amount of data present and individually picking whether an image contains a fault or not would take a long time. As stated earlier, the true fault model exists to compare with the migrated images. The same subset of the migrated images can be compared with the reflectivity model. If more than half the pixels are a fault, then the program labels the training data as a fault (Figure~\ref{fig:HorizonExamples}). Otherwise, the program labels the training data as not a fault (Figure~\ref{fig:NotHorizonExamples}).

\multiplot{2}{HorizonExamples,NotHorizonExamples}{width=.6\textwidth}{(a) Examples of the automatically generated faults images used to train the CNN. (b) Examples of the automatically generated images that were not faults used to train the CNN.}

This is an easy and automatic way to generate training data, but training is an essential step prior to testing, so it needs to be continually improved. The next step is to QC the training data to make sure that the examples are actually of ``faults" and ``not faults". There is a lot of back and forth until an acceptable cross-validation accuracy is achieved. A total of 2500, 100 meter by 100 meter windowed RTM images were used to train the CNN to detect faults. A final training validation accuracy of 94.4\% is achieved. This is an acceptable accuracy check and now the neural network is ready to be tested on data that were not included in the training data.

A 100 meter by 100 meter testing data is created the same way the training data is created. The testing data is kept hidden from the training data. The first RTM image that is used for testing is the vertical source data from the velocity model shown in Figure~\ref{fig:dtestVP}. The first test is on the sparse, multi-component geophone image (Figure~\ref{fig:dtestEIMGS}). The RTM image is decomposed into 3,625 (100 meter x 100 meter) images with labels of ``Faults" and ``Not Faults". A confusion matrix is then calculated by evaluating the predictions to the actual fault or no fault classifications of the test data. This same process is used for the synthetic created from DAS and multi-component geophones. The results of these experiments are shown in Table~\ref{table:CNNResults-dtest}.

\begin{table}[]
\centering
\caption{Confusion matrices for CNN created from the geophones (Figure~\ref{fig:dtestEIMGS}) and both data types together (Figure~\ref{fig:dtestTot}) using a vertical force source.}
\label{table:CNNResults-dtest}
\setlength{\tabcolsep}{1em}
\begin{tabular}{|l|l|l|ll|l|l|l|}
  \multicolumn{3}{c}{Multicomponent Geophone} & \multicolumn{2}{c}{ } &\multicolumn{3}{c}{DAS \& Geophone}\\
  \hline
  & $\theta_F^{int}$ & $\theta_{NF}^{int}$ & & & &$\theta_F^{int}$ & $\theta_{NF}^{int}$ \\
\hline
$\theta_F$  & 292 & 1008 & & & $\theta_F$   & 420 & 1870\\
\hline
$\theta_{NF}$ & 327 & 1998 & & & $\theta_{NF}$ & 199 & 1176 \\
\hline
\end{tabular}
\end{table}

Now that a confusion matrix of results is available, a posterior reliability of information can be calculated just as it was completed for the energy norm filtered images. The resulting posterior reliability of information is shown in Table~\ref{table:POSTERIORS-vSource-CNN}.

\begin{table}[]
\centering
\caption{Posterior reliability of information from CNN calculated using Equation~\ref{eqn:Posterior} using a vertical force source with Figure~\ref{fig:dtestVP} as the velocity model.}
\label{table:POSTERIORS-vSource-CNN}
\setlength{\tabcolsep}{1em}
\begin{tabular}{|c|c|c|}
  \hline
  &  Vertical  & Vertical  \\
  &  Source &  Source   \\
  &  Geophone      & Geophone  \\
  &                   &   \& DAS \\
  \hline
  $Pr(\theta=\theta_F | \theta^{int}=\theta_F)$ &   47.17 \% & 67.85  \%\\
  \hline
  $Pr(\theta=\theta_F | \theta^{int}=\theta_{NF})$ &  33.53 \% & 61.39 \%  \\
  \hline
  $Pr(\theta=\theta_{NF} | \theta^{int}=\theta_F)$ &  52.83 \% &  32.14 \%  \\
  \hline
  $Pr(\theta=\theta_{NF} | \theta^{int}=\theta_{NF})$ & 66.47 \% & 38.61 \% \\
  \hline
\end{tabular}
\end{table}

The results from Table~\ref{table:POSTERIORS-vSource-CNN} for the vertical source shows that adding DAS into the sparse array of geophones with \ref{fig:dtestVP} as the velocity model improves the classification of faults by 20\%. However, there is an increase in false negatives by about 30\%. This means either the normalized, stacked image has many artifacts or the classifier needs to be better trained on what is not a fault. The number of false positives decreases by 20\% which is a substantial amount. Lastly, the number of true negatives decreases by almost 30\%. This confirms that the classifier needs to be better trained on what is not a fault.

Although it was overtrained on true faults, the CNN classifier does a much better job at classifying an image than energy norm filter. The false positives and false negatives will decrease with more training iterations especially in training the classifier on images that are not faults.

\subsection{Summary}
This section discussed in great detail how 2-D DAS data can be modeled. It also showed how a long offset, 2-D surface DAS line can produce a sharp resulting image. A quantitative analysis using two methodologies showed that DAS does add value to sparse geophone arrays. This hypothesis must now be confirmed with a 3-D acquisition.

\section{3D Numerical Modeling Examples}
The objective of this paper is to observe if there is any added value of using surface DAS with sparsely \sout{sampled} \hl{arranged}, multi-component geophones. In the previous section, we observed that in a long 2D line, there is added value using DAS to help with the spatial sampling. In 3D, however, there are many more complications than in 2D. This section explores additional examples of using DAS in combination with multi-component geophones, but now with the PoroTomo 3D survey geometry. These examples utilize numerical modeling to understand more about what is recorded.

A velocity model from sweep interferometry shown in Figure~\ref{fig:EricModel-Mean} was used to create data \citep{matzelseismic}. As in the previous section, a modified version of the conventional elastic FDM code (ewefdm) present in Madagascar \citep{fomel2013madagascar} is utilized, but now for the 3D case. This allows us to recover both displacement and strain data along receivers in the grid. A variable density is now used to create reflectivity instead of using purely velocity changes to create reflectivity in the 2D case.

\plot{EricModel-Mean}{width=\textwidth}{Body wave (P-wave) velocity (m/s) model from sweep interferometry in 3D perspective \citep{Matzel2017}.}


\subsection{3D Modeling of Non-Uniform DAS Acquisition}
The wavefield along the fiber is now recorded for the six components of strain (XX, XY, XZ, YY, YZ, and ZZ). Field DAS data with single fiber, however, does not recover all six components. Instead, it only recovers contributions of the wavefield in the direction that it is oriented. We can project the six components from the synthetic data on to the vector direction of the field fiber locations to recover the strain in the direction that the fiber is oriented by using Equation~\ref{eqn:StrainOneComponent},

\begin{equation}
\begin{bmatrix}
  \varepsilon_{ZZ} \\
  \varepsilon_{XX} \\
  \varepsilon_{YY} \\
  \varepsilon_{XY} \\
  \varepsilon_{YZ} \\
  \varepsilon_{ZX} \\
\end{bmatrix}
\begin{bmatrix}
  V_Z^2 & V_X^2 & V_Y^2 & 2V_XV_Y & 2V_YV_Z & 2V_ZV_X
\end{bmatrix}
=
\begin{bmatrix}
  \varepsilon'
\end{bmatrix}
\label{eqn:StrainOneComponent}
\end{equation}

where $\varepsilon_{ij}$ is the strain in the direction $ij$, $V_{i}$ is the vector projection in the $i$ direction, and $\varepsilon'$ is the strain in the direction of the fiber.

A matrix of fiber vector directions must be created prior to using Equation~\ref{eqn:StrainOneComponent}. The fiber endpoints were recorded in the field using a handheld GPS device after the fiber was trenched. The virtual receiver locations along the fiber were then interpolated at 1-meter spacing between these endpoints. Although this gives a good estimate of the x and y coordinates of the fiber, this does not give any information on how deep the fiber was trenched. For this reason, we assume that the fiber was all trenched in the same horizontal plane and there are no dips along the fiber. This simplifies Equation~\ref{eqn:StrainOneComponent} to only have contributions from X and Y.

Applying Equation~\ref{eqn:StrainOneComponent} recovers only one value of strain along the fiber. In reality, there are contributions from both X and Y, so the strain matrix should have values at XX, YY, and XY. We can use the adjoint operation to recover a vector projection of the strain value from Equation~\ref{eqn:StrainOneComponent}. The adjoint operation shown in Equation~\ref{eqn:StrainMultiComponent} returns back to the original PoroTomo coordinate system.

\begin{equation}
  \begin{bmatrix}
    \varepsilon'
  \end{bmatrix}
\begin{bmatrix}
  V_Z^2 \\ V_X^2 \\ V_Y^2 \\ 2V_XV_Y \\ 2V_YV_Z \\2V_ZV_X
\end{bmatrix}
=
\begin{bmatrix}
  \varepsilon_{ZZ} \\
  \varepsilon_{XX} \\
  \varepsilon_{YY} \\
  \varepsilon_{XY} \\
  \varepsilon_{YZ} \\
  \varepsilon_{ZX} \\
\end{bmatrix}
\label{eqn:StrainMultiComponent}
\end{equation}


The last attribute of fiber that needs to be modeled is the gauge-length. As discussed earlier in this paper, the gauge-length of fiber is related to the wavelength recorded along the fiber and it acts as a moving average. The gauge-length in the PoroTomo survey was set at 10-meters, so the modeled data, $d$, is a matrix multiplication of $\frac{1}{10}$ for the gauge length, the spatial sampling 1-meter, and the raw point data, $b$, recorded by the finite difference code (shown in Equation~\ref{eqn:GaugeLength}, after Lim Chen Ning and Sava, 2018).

\setcounter{MaxMatrixCols}{20}
\begin{equation}
\begin{bmatrix}
d_5 \\
d_6 \\
d_7 \\
\vdots \\
d_{n-5}
\end{bmatrix}
=
\frac{1}{10}
\begin{bmatrix}
1 & 1 & 1 & 1 & 1 & 1 & 1 & 1 & 1 & 1 & 0 & 0 & 0 & 0 \\
0 & 1 & 1 & 1 & 1 & 1 & 1 & 1 & 1 & 1 & 1 & 0 & 0 & 0 \\
0 & 0 & 1 & 1 & 1 & 1 & 1 & 1 & 1 & 1 & 1 & 1 & 0 & 0 \\
0 & 0 & 0 & \ddots  & \ddots & \ddots & \ddots & \ddots & \ddots & \ddots & \ddots & \ddots & \ddots & 0\\
0 & 0 & 0 & 0 & 1 & 1 & 1 & 1 & 1 & 1 & 1 & 1 & 1 & 1
\end{bmatrix}
\begin{bmatrix}
  b_5 \\
  b_6 \\
  b_7 \\
  \vdots \\
  b_{n-5}
\end{bmatrix}
\label{eqn:GaugeLength}
\end{equation}

% The DAS data are now ready to be modeled as they are recorded in the field after all of the steps described in this section.

\subsection{Numerical Modeling}
It is important to image the faults in detail at Brady's Natural Lab as they are the driving factors behind the recharge of the geothermal reservoir. Although \citet{siler2013three} would be a good candidate for data modeling, a simpler model is needed to first test the hypothesis of imaging using the two data types simultaneously. A four layer model with a variety of structures is used as the density model for the first example (Figure~\ref{fig:den3D}). There is a contrast of about 300 g/cc between each layer to ensure strong reflections.

\plot{den3D}{width=\textwidth}{Four layer model with a variety of structures used for data modeling. This model is used as a density model for elastic modeling.}
%
% The first step for this synthetic experiment is to generate data. The data is generated from the reflectivity model. The reflectivity in this experiment is caused by the density contrast in the model subsurface. The variable density, elastic FDM operator is used to propagate a wavefield through the reflectivity model from a given source location. The strain data is recorded along the fiber and the displacement data is recorded at the geophones.
%
% The next step is to create a receiver wavefield. The receiver wavefield is conventionally recovered by back propagating the recorded field data. In the case of this numerical experiment, the data that were synthetically generated from a reflectivity model are back propagated. The data are reversed in time, reinjected at all the receiver points, and propagated using the forward elastic operator. Again, two different sources are needed to create the receiver wavefield. An acceleration force source is used for back propagation of the geophone data and a stress tensor source is used for back propagation of the DAS data. There is no current way to create both sources at the same time in the Madagascar package. Instead, two different receiver wavefields are created for the DAS and geophone data. Now, a source and two receiver wavefields exist. An imaging condition is required to combine the two wavefields.
%
% The 2D section first discusses the option of using the conventional imaging condition by \citet{claerbout1985imaging} which involves taking the zero-lag, cross correlation at each time step for every experiment. This IC can be extended into 3D space with the additional y-component of the wavefield. It is difficult to make comparisons with all of the images that come out of the conventional elastic IC. Instead, the energy norm IC is utilized to produce one final image as discussed in the 2D numerical modeling section \citep{rocha2016isotropic}.

The synthetic images are produced using the same methodology presented in the 2D section. The results from migrating the DAS data are shown in Figure~\ref{fig:DasRTM-3Dacq}. The results from migrating the geophone data are shown in Figure~\ref{fig:GeophoneRTM-3Dacq}. A visual reflectivity model shown on the left of both figures was produced by applying the Laplacian operator on Figure~\ref{fig:den3D} and setting all values to one.

\plot{DasRTM-3Dacq}{width=\textwidth}{Results of migrating the reflectivity model (Figure~\ref{fig:den3D}) using the PoroTomo DAS acquisition shown on the right. The true reflectivity model is overlain and shown on the left. The slices on each side are taken at the yellow cross shown on the map view of the acquisition. }

\plot{GeophoneRTM-3Dacq}{width=\textwidth}{Results of migrating the reflectivity model (Figure~\ref{fig:den3D}) using the PoroTomo geophone acquisition shown on the right. The true reflectivity model is overlain and shown on the left. The slices on each side are taken at the yellow cross shown on the map view of the acquisition. }

At first glance, it seems as if the DAS image does not have any reflectors. It can be compared to the true reflectivity model shown on the right of Figure~\ref{fig:DasRTM-3Dacq} to identify the \sout{signal} \hl{true horizons} in the image. It is clear that the data recorded by the DAS fiber is too low in frequency to resolve the beds within the image. This is due to both the velocity field that the experiment used to mimic the PoroTomo subsurface and the FDM accuracy condition presented in Equation~\ref{eqn:accuracy3D}.

\begin{equation}
  \frac{v_{min}}{f_{max}} > N * \sqrt{dx^2+dy^2+dz^2}
\label{eqn:accuracy3D}
\end{equation}

The minimum velocity of approximately 950 m/s from the input velocity field forces the maximum frequency of the wavelet to be 16 Hz and the peak frequency of the wavelet to be 12 Hz. This equates to a 12 Hz wavelet and the velocity model corresponds to a wavelength of about 108 meters.

The DAS image (Figure~\ref{fig:DasRTM-3Dacq}) is also contaminated by fake modes and migration artifacts \citep{rocha2016isotropic}. Fake modes are expected since the displacement field is incomplete when wavefield extrapolation was performed as the fiber is only recording one component of strain in the direction that it is oriented. An inexperienced interpreter would eagerly interpret the fake modes as an area of interest for further exploration methods.

At first glance, the geophone data also appear to have no clear reflection events. The image can again be compared to the true reflectivity model overlain on the left of Figure~\ref{fig:GeophoneRTM-3Dacq} to identify the \sout{signal} \hl{true horizons} in the image. The geophone image is also limited by the source wavelet that was injected into the model. Differentiation between the thin beds is not possible using the source wavelet in this experiment.

The geophone image, similar to the DAS image, is also contaminated by migration artifacts. These migration artifacts, however, are due to the insufficient sampling that creates migration artifacts on the edge of reflectors. The wavefield is not sampled completely because the geophones adopted from the PoroTomo survey are placed sparsely around the model (the average geophone spacing is about 80 meters).

\subsection{Quantitative Image Comparison}
In 2D, a machine learning methodology was used to create a quantitative image comparison. Although 3D CNN's exist, they are not as polished and readily available as are 2D CNN's. Instead, the data are quantitatively analyzed using energy norm image filtering. Energy norm filtering focuses on highlighting areas with reflected energy is maximum, so filtering the image based on an applied limit will highlight where reflections may be coming from as opposed to migration artifacts. The geophone and DAS images are combined by first normalizing the data types based on their maximum amplitude. They are then stacked together to test this hypothesis. This image would ideally highlight continuous reflectors with the densely sampled DAS data and reduce migration artifacts by extrapolating the full displacement wavefield with the multi-component geophones.

Every model cell that is above an applied limit is assigned a value of 1 and every model box that is below the limit is assigned a value of 0. A cell-by-cell comparison between the filtered, multi-component geophone image and the original reflectivity model is performed to identify how much additional accuracy is gained by adding the DAS data. The results of this cell-by-cell comparison are presented in confusion matrix form (Table~\ref{table:3dCONF}), where $R$ represents reflections and $NR$ represents not reflections.

% The term $\theta_R$ denotes that an actual reflector exists and the term $\theta_{NR}$ denotes that no reflector exists.

% The terms $\theta_R^{int}$ and $\theta_{NR}^{int}$ represent the interpretations of reflector and no reflector, respectively, based on energy filtering of images. The columns of the confusion matrix represent predicted classifications ($\theta_R^{int}$ and $\theta_{NR}^{int}$). The rows of the confusion matrix represent actual true statement of the subsurface ($\theta_R$ and $\theta_{NR}$).

\begin{table}[]
\centering
\caption{Confusion matrix for top 90\% energy reflected.}
\label{table:3dCONF}
\setlength{\tabcolsep}{1em}
\begin{tabular}{|c|c|c|}
 \multicolumn{3}{c}{Top 90\% energy reflected} \\
  \hline
 & $\theta_R^{int}$ & $\theta_{NR}^{int}$ \\
            \hline
 $\theta_R$ & 184800 & 1206000  \\
\hline
$\theta_{NR}$ & 346700 & 1824000 \\
\hline
\end{tabular}
\end{table}

The confusion matrices assist in calculating the posterior value using Equation~\ref{eqn:Posterior}. The posterior value explains the probability that an event which the data type predicted is the event present. The posterior can then be used to calculate the utility or value of information added when using DAS and geophone versus only geophone with Equation~\ref{eqn:VOI}. The results for the medium filter, posterior values in the four layer model presented in this paper are displayed in Figure~\ref{fig:3DenormPOSTERIORS-hsource}.

\plot{3DenormPOSTERIORS-hsource}{width=\textwidth}{Posterior reliability of information from energy norm filtering calculated using Equation~\ref{eqn:Posterior} using a horizontal force. The objective is to maximize the percentages of true positives and negatives (green arrows) while minimizing the percentages of false positives and negatives (red arrows). }


% \begin{table}[]
%   \centering
% \caption{Posterior reliability of information calculated using Equation~\ref{eqn:Posterior} for a 90\% energy filter.}
% \label{table:POSTERIORS-4layer}
% \setlength{\tabcolsep}{1em}
% \begin{tabular}{|l|l|l|}
%   \hline
%   & Horizontal  & Horizontal    \\
%   &  Source  &  Source 3C \\
%   &    3C &       Geophone     \\
%   & Geophone   &   \& DAS      \\
%   \hline
%   $Pr(\theta=\theta_R| \theta^{int}=\theta_R)$ & 34.77 \% & 34.7 \% \\
%   \hline
%   $Pr(\theta=\theta_R | \theta^{int}=\theta_{NR})$ & 39.79 \% & 39.64 \% \\
%   \hline
%   $Pr(\theta=\theta_{NR} | \theta^{int}=\theta_R)$ & 65.23 \% & 65.3 \%  \\
%   \hline
%   $Pr(\theta=\theta_{NR} | \theta^{int}=\theta_{NR})$ & 60.21 \% & 60.39 \%  \\
%   \hline
% \end{tabular}
% \end{table}

In this experiment, adding distributed sensors increases the probability of finding if a cell is not a reflector and decreases the probability of false negatives. Adding distributed sensors, however, increases the probability of identifying false positives and decreases the probability of finding true reflectors. This experiment, however, is inconclusive in identifying if DAS has added value with sparsely \sout{sampled} \hl{arranged} geophone data. A better DAS geometry must be tested to make further conclusions on the effectiveness of surface DAS fiber.

The geometry shown in Figure~\ref{fig:idealstr8} is utilized to further test the effectiveness of surface DAS fiber. This new acquisition utilizes 25\% less fiber and 60\% fewer sources than the PoroTomo survey geometry. Quantitative analysis using the energy norm filtering methodology is utilized again to identify how well the survey imaged. The results are presented in Figure~\ref{fig:3DenormPOSTERIORS-str8}

\plot{idealstr8}{width=.7\textwidth}{New survey geometry proposed to test the effectiveness of surface DAS fiber. Green dots represent source locations, red dots represent geophone locations, and the blue lines represent the surface DAS acquisition.}

\plot{3DenormPOSTERIORS-str8}{width=\textwidth}{Posterior reliability of information using a horizontal force and the Figure~\ref{fig:idealstr8} acquisition geomtery. }

Figure~\ref{fig:3DenormPOSTERIORS-str8} shows a significant increase in true positives and decrease in false negatives. Although there was an increase in false positives and a decrease in true negatives, the increase in true positives proves that this new acquisition is better suited to image the subsurface with surface DAS fibers. Energy norm imaging again allowed for an automatic method to interpret images output from the migration images. Filtering images based on amplitudes is a crude approximation of how an interpreter would ``interpret" an image.

\subsection{3D Summary}
This section discussed differences in modeling DAS data in 3D versus 2D. The experiments in this section helped clarify what kinds of data that a single surface DAS fiber can record. The experiments discovered that the DAS configuration in the PoroTomo survey combined with the low frequency nature of the modeling did not add value to the multi-component geophone imaging effort. Additionally, the percentages of missing strain components in 3D is larger than the 2D case, contributing to the poor image quality. A better geometry and multi-component DAS were required to make further conclusions on the effectiveness of DAS fiber in surface acquisition. Another experiment was preformed with DAS fibers arranged in 2D lines. This acquisition geometry led to an increased percentage of reflectors identified. It is concluded that the 2D surface DAS fiber lines are a better suited geometry to image the subsurface.

\bibliographystyle{seg}
\bibliography{demobib.bib}
