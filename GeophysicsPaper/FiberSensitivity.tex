\section{Fiber Sensitivity}
Understanding how DAS fiber works is essential to working with the data that is currently available and for future surveys. For a conventional DAS seismic survey, a known pulse of light is sent into the fiber using an interrogator unit and some of the light is naturally scattered back due to imperfections within the fiber. The interrogator unit is able to record this scattered light along the fiber up to 10-kilometers away. This is known as the base condition inside of the fiber. The fiber undergoes a strain when a seismic wavefield approaches and a scattering of light is produced that is different from the base condition. The interrogator unit is able to relate this new scattering of light to local strain along the fiber by recording the time of arrival and the phase-lag of the returning light signals \citep{Parker2014}.

As stated previously, DAS fiber is most sensitive to waves that are able to stretch and squeeze the fiber, so the waves have to have particle motion parallel to the orientation of the fiber. Every seismic sensor has its own distinct sensitivity to the various types of waves depending on their emergent angle. The emergent angle ($\theta$) represents the angle between the incoming wave and the surface of the Earth. Consider a plane wave reflection in the X-Z plane: an emergent angle of $0^{\circ}$ represents a wave arriving parallel to the surface (or a plane wave traveling in the Z-direction); an emergent angle of $90^{\circ}$ represents a wave arriving perpendicular to the surface (or a plane wave traveling in the X-direction).

These points can be demonstrated with a simple 2-D example. Consider wave propagation in the x-z plane in a homogeneous, flat-layered, isotropic or vertical transverse isotropic medium (Figure~\ref{fig:Preflection2}-\ref{fig:SVReflection2}). The horizontal DAS fiber is oriented in the x-direction. P-waves have particle motion parallel to the direction of wave propagation \citep{aki1980quantative}. Normal-incidence reflections from a horizontal reflector will arrive perpendicular to the surface fiber.  In the case of a 2-D line of horizontal fiber with an vertical vibe, the reflected P-wave will not be seen at short offsets (Figure~\ref{fig:Preflection2}). The particle motion of P-waves is parallel to the direction of propagation, so at short offsets, the reflected P-wave will arrive perpendicular to the fiber. As seen in Figure~\ref{fig:TotPwave}, P-waves with a $0^{\circ}$ emergent angle show 0 amplitude on fiber and maximum amplitude on the z-component of a geophone. Moving to further offsets yields emergent angles that are at a larger angle to the fiber. According to Figure~\ref{fig:TotPwave}, these waves will show more data than waves arrive perpendicular to the fiber as they are propagating in the direction of the fiber and less data on the vertical component of the geophone. The further the offset, however, the lower the amplitude of the wave due to attenuation effects.

Shear-waves are potentially more interesting when recording with horizontal fiber. Consider again 2-D wave propagation in the x-z plane in a homogeneous, flat-layered, isotropic or vertical transverse isotropy medium (Figure~\ref{fig:SVReflection2}). SV-waves have particle motion in the x-z plane, as do P-waves \citep{aki1980quantative}. Normal-incidence reflections from a horizontal reflector will arrive perpendicular to the surface fiber. P-wave particle motion, as stated previously, will be in the z direction, and consequently, will not be recorded by the fiber. SV wave particle motion will be in the x-direction (emergent angle of $0^{\circ}$), and the DAS response will be maximum (Figure~\ref{fig:TotSwave}). At larger offsets, the SV-wave emerging angle begins to approach $90^{\circ}$. A larger emerging angle means less signal (Figure~\ref{fig:TotSwave}) is recorded by both the surface DAS and the x-component of the geophone because the SV-wave particle motion approaches perpendicular to these components (Figure~\ref{fig:SVReflection2}).

\multiplot{2}{Preflection2,SVReflection2}{width=\textwidth}{Consider a source that generates both P and S waves; this is a 2-D Homogeneous, flat-layered, isotropic ray  path example. The blue lines represent ray paths of the labeled waves, the blue arrow represents the propagation direction of the wave, the green line represents a horizontal reflector in the subsurface, and the yellow star represents the source. (a) Demonstration of P-P wave effect on fiber using ray paths. Particle motion is inline with propagation direction (blue arrow). The fiber will only record data at large offsets. (b) Demonstration of P-SV wave effect on fiber using ray paths. Particle motion is perpendicular with the propagation direction (green arrows). The fiber will only record data at short offsets.}

\multiplot{2}{TotPwave,TotSwave}{width=\textwidth}{Amplitude versus emerging angle for both geophone (blue) and horizontal DAS (red). An emerging angle of $0^{\circ}$ indicates a wave that is propagating perpendicular to the surface and an emerging angle of $90^{\circ}$ degrees indicated a wave is propagating parallel to the surface. (a) Sensitivity with respect to a P-wave at different emerging angles of horizontal DAS (red) and geophone z-component (blue). (b) Sensitivity with respect to an SV-wave at different emerging angles of horizontal DAS (red) and geophone x-component (blue).}

We also consider using SH-waves with the same 2-D survey geometry (homogeneous, flat-layered, isotropic). SH-waves propagate in the x-z plane. SH waves have particle motion perpendicular to the direction of wave propagation or, in this case, in the y-direction. The SH-DAS response will be zero since the SH particle motion is perpendicular to the DAS fiber in the y-direction. In this 2-D case, the SH-wave will be out of plane regardless of source-receiver offset. In 3-D, SH-waves can be seen on the DAS if they are properly oriented. For example, a source-receiver azimuth perpendicular to the 2-D fiber (in this case, in the y-direction) will produce a maximum amplitude reflection on the DAS since the particle motion is in the x-direction for all offsets. As the source-receiver azimuth moves inline with the fiber, the SH-wave particle motion decreases, and is equal to zero when the source-receiver azimuth is inline with the fiber.

% In this section, DAS fiber directionality was described analytically and depicted graphically. The directionality is important to understand when creating a seismic survey geometry.
