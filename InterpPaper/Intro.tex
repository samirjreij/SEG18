
\section{Introduction}
Distributed acoustic sensing (DAS) is a technology that uses Rayleigh scattering in a fiber-optic cable to detect elastic \sout{signals} \hl{data} when wave particle motion is parallel to the sensing fiber \citep{hornman2013field}. The two main components used in distributed sensing are the interrogator unit and the fiber-optic cable. The interrogator unit works as a light source and receiver. It sends a known pulse of light down the fiber. Nearly any type of existing fiber-optic cable can be used in conjunction with an interrogator unit. Small imperfections within the fiber cause backscattering of light. Strain events along the fiber cause this backscattering to change slightly when a wavefield approaches the fiber. The interrogator unit can measure the Rayleigh backscattering and relate it to the strain along the fiber.

\subsection{DAS Advantages}
DAS has many advantages in various industries. For one, DAS is a low-cost acquisition system in wells that already contain fiber optic cables \hl{ \citep{mateeva2014distributed} }. \sout{Even in those wells that do not already contain fiber optic cables, a DAS vertical seismic profile (VSP) is often more affordable than renting and deploying geophones \citep{mateeva2014distributed}.} \hl{Even in those wells that do not already contain fiber optic cables, a DAS vertical seismic profile (VSP) may be more affordable than renting and deploying geophones for a full-well VSP.} DAS also enables seismic surveys to be acquired with dense sampling (as small as 10-centimeter receiver spacing) at long cable lengths (tens of kilometers long). Achieving even 1-meter sampling with conventional geophone is expensive and logistically difficult. \sout{Lastly, DAS has almost perfect repeatibility in 4-D surveys when cemented in a borehole, attached to casing, or trenched in the subsurface \citep{mateeva2013distributed}.} \hl{Lastly, DAS VSP repeatability in 4-D surveys is straightforward and economical when permanently installed \citep{mateeva2013distributed}. This can be carried over to surface surveys as permanent installation is possible in trenches.}

\subsection{DAS Disadvantages}
Although DAS may seem like the solution to seismic acquisition, it also has many disadvantages. DAS is most sensitive to waves that have particle motion parallel to the orientation of the fiber, so it is said that the technology has broadside insensitivity.  Multi-component geophones also have this issue; with more recording components, however, they are able to resolve more of the wavefield and are not affected by this broadside insensitivity as much.

Another disadvantage is that DAS coupling is not trivial in all environments. In a borehole environment, DAS can be cemented behind casing or permanently installed on production tubing \citep{mateeva2013distributed}. Surface distributed sensor coupling is a more challenging issue. \citet{lindsey2017fiber} describe how fibers can be utilized in loosely coupled environments. The Stanford Fiber Optic Array consists of a 2.5 km long array that lies in a conduit about 1 to 2 meters below ground. The DAS fiber geometry is restricted by the conduits, though, and the task of installing the fiber is more difficult if there are no existing conduits. \citet{daley2013field} have trenched the cable and returned at a later time to shoot the seismic survey. Although this method is effective, waiting to shoot a survey at a later time can be inconvenient.

\subsection{Previous Work}
Historically, DAS has been used in a borehole environment for flow monitoring, temperature measurements, and vertical seismic profiles \citep{clarke1983fiber, krohn2000fiber,mestayer2011field, barberan2012multi, cox2012distributed, daley2013field, mateeva2014distributed}.

\citet{mestayer2011field} describe how permanently installed fiber-optic infrastructure in existing wells can enable low-cost non-intrusive geophysical monitoring. Geophones generally only acquire data along a short subset of the well due to the limited number of receivers at predetermined receiver spacing in VSP receiver arrays. This makes repeatable time-lapse surveys difficult as placing the geophones in the same location is not trivial. \citet{mestayer2011field} also discuss how borehole DAS is able to improve repeatibility and time-lapse sensitivity because it is able to acquire data along the full well with a single shot. \citet{mateeva2014distributed} also describe a time lapse, 3D DAS VSP application. They conclude that DAS has many major business impacts on fields that require enhanced oil recovery (EOR) including cost efficiency, safety, and synergy with other fiber optic applications.

\citet{barberan2012multi} discuss different ways DAS fiber can be coupled in a borehole environment. DAS fiber can be clamped to production tubing and used as a downhole seismic sensor. \citet{barberan2012multi} expand on this explaining that acquiring seismic data over the entire well is essential for acquiring additional transit times for velocity inversion and it allows for a wide range of incidence angles in terms of wave directions that arrive at the fiber for inversion.

% \citet{cox2012distributed} also looks at borehole DAS data, but now for the purpose of imaging. \citet{cox2012distributed} examine images produced from both geophone and DAS data types in a borehole example. They conclude that images produced from distributed sensing are similar to those produced by geophones and that DAS has potential to replace geophones.

\citet{daley2013field} describes field tests from both horizontal and borehole distributed sensors. They conclude that the signal-to-noise (SNR) in surface DAS is not sufficient for observing P-waves and that DAS is more useful in borehole environments or longer surface arrays. They don't go into detail, however, regarding why certain waves are not observable in DAS.

As seen in these examples, DAS research has emphasized acquiring data in borehole environments because many wells are already equipped with fiber for production. As a result, acquiring DAS in boreholes is as simple as connecting the existing fiber-optic cable to a new interrogator unit that senses acoustic \sout{signal} \hl{data}. Although there are some studies on surface DAS acquisitions \citep{daley2013field, hornman2017field}, there has not been a thorough study in active source experiments.

% Most recent papers in DAS fiber focus on how the system can be used independently of geophones. This is viable in a borehole environment due to the fact that the predominant particle motion of the recorded seismic data is in the vertical direction, parallel to the fiber. Surface horizontal DAS is sensitive to the horizontal component of particle motion. P-wave reflections will not be recorded on surface DAS at normal incidence, assuming a flat-layered earth, since the particle motion is vertical.

\citet{daley2013field} experiment with a \sout{v}vertical vibrator (vertical-force) source. The reflected P-wave is not recorded on the DAS fiber as the experiment only had 1,000 meters of offset, and, therefore, the authors concluded that the SNR in surface DAS is insufficient for observing P-waves due to the relatively small incidental reflected angle. Other source mechanisms must be investigated before such a conclusion can be made about the feasibility of using surface DAS fiber. Another option is utilizing the DAS fiber along with geophones to attempt to minimize the insensitivity of some waves. This paper explores different imaging experiments using the field geometry from the PoroTomo survey in Northwest Nevada and numerical modeling to explain how DAS fiber can help minimize the insensitivity to waves in conjunction with geophones. The objective of these experiments is to analyze if the densely sampled DAS fiber data can help improve the image produced by the sparsely sampled geophones.

\subsection{PoroTomo Survey}
The PoroTomo survey involved four-weeks of data acquisition of geodesy, interferometric synthetic aperture radar (InSAR), hydrology, temperature sensing, passive source seismology, and active source seismology data \citep{feigl2017overview,cardiff2018geothermal}. The variety of data that were collected at the PoroTomo survey lead to the origin of the experiments name: Poroelastic Tomography by Adjoint Inverse Modeling of Data from Seismology, Geodesy, and Hydrology (or PoroTomo for short). These data were jointly collected to characterize and monitor changes in the rock mechanical properties of Brady's Natural Laboratory (BNL), an Enhanced Geothermal System (EGS) reservoir.

This paper investigates the active seismic source component of the PoroTomo Experiment. The PoroTomo survey is one of the most unique seismic acquisitions for surface DAS fiber. The survey included 238 multi-component geophones, 156 three-component (vertical and orthogonal horizontal) vibroseis source locations that swept from 5 to 80 Hz in 20 seconds, 300 meters of borehole DAS, and nearly nine kilometers of surface fiber-optic cable. The full survey geometry is shown in Figure~\ref{fig:dasngeophone}. As seen in Figure~\ref{fig:dasngeophone}, the geophones are sparsely spaced with an average inline spacing of 80 meters. This paper focuses on identifying a methodology to resolve the spatial sampling issue. The objective of this paper is to identify if the densely sampled DAS data can help improve the image produced by the sparsely sampled geophones. Both 2D and 3D numerical experiments are performed to test the feasibility of using the broadside sensitivity of multi-component geophones and the dense sampling DAS data together to minimize insensitivity to certain waves.

\plot{dasngeophone}{width=\textwidth}{PoroTomo survey geometry. Green dots represent source locations, red dots represent geophone locations, and the blue line represents the surface DAS layout.}
%
% \subsection{Outline}
% This paper explains, and quantitatively analyzes, different approaches for using surface DAS in active source seismic acquisitions. First, the theory behind DAS fiber and  wave types that DAS fiber can record is introduced using theoretical examples. Then, 2D numerical modeling is utilized to perform a feasibility study on combining DAS and multi-component geophones data to create one image. This is theoretically useful as the multi-component geophones sample the full wavefield (Z, X, and Y components of particle motion) at sparse locations and the DAS fiber samples one component of the wavefield densely. The 2D examples shown are idealized experiments. More extreme examples must be performed in 3D prior to making any conclusions with the hypothesis previously presented, so the challenges with modeling DAS are discussed and a numerical modeling example is presented to test if adding DAS data can improve a multi-component DAS image in 3D.
