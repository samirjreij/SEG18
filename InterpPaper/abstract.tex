\begin{abstract}

  The PoroTomo survey at Brady's Natural Lab consisted of 238 multi-component geophones that are spaced anywhere from 60 to 150 meters apart. This proves to be a difficult migration problem with such sparse spacing. An imaging technique that utilizes both multi-component geophones and a surface distributed acoustic sensor (DAS) acquisition attempts to resolve the spatial sampling issue. Fortunately, the PoroTomo survey consisted of surface DAS cable with a 1-meter receiver spacing along the fiber. Both 2D and 3D numerical experiments test the feasibility of using the broadside sensitive multi-component geophones and the densely sampled DAS data together to minimize insensitivity to certain waves. The objective of these experiments are to analyze if the densely sampled DAS fiber data can help improve the image produced by the sparsely sampled geophones. In 2D, a reflectivity model is created from the local fault model in the PoroTomo Survey. Quantitative analysis provides an unbiased comparison of the results. The quantitative analysis utilizes a convolutional neural network to prove that DAS adds value to imaging efforts. A more challenging example in 3D confirms the conclusions made in 2D. A methodology to model DAS data in 3D shows that utilizing DAS in surface surveys with sparse, multi-component geophones proves to be useful in improving the classification accuracy of the image. The results, however, are inconclusive because the migrated images are too low of a frequency to analyze due to the limitation of the velocity model. Lessons learned from the data collected at the PoroTomo survey and the numerical experiments are that a more regular acquisition geometry of the horizontal DAS fiber increases identifying the true reflectors.

\end{abstract}
