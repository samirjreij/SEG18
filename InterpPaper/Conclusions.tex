\section{Conclusions}
In this paper, we discussed some of the fiber attributes that are essential to understand before looking at surface DAS data. The most important attribute is the types of waves that fiber is directionally sensitive. It is concluded that surface DAS in a flat-layered Earth model is sensitive to long offset P-waves, short offset SV-waves, and SH-waves produced by a source that is perpendicular to the fiber orientation.

It was discovered that the geophone data in the PoroTomo survey was too sparsely \sout{sampled} \hl{arranged}, and the hypothesis that \sout{densely sampled DAS data} \hl{DAS, which samples the Earth's response well,} can fill in the gaps of the geophones was introduced. This hypothesis was tested in 2D using elastic numerical modeling and RTM. It is then shown how data are modeled for DAS receivers in 2D. The energy norm imaging condition was chosen as it allowed for an easier method to compare two images than the conventional imaging condition. The experiments showed that an inline horizontal force allows for the best results qualitatively due to the resulting SV reflections. Lastly, this section explains the need for statistical and quantitative analysis in the geophysics realm. A description of how to perform quantitative analysis using machine learning methodology is presented. Both methods concluded that DAS added imaging value to sparsely sampled multi-component geophones.

The 2D scenario did not test the full DAS fiber directionality. We discuss 3D numerical modeling and RTM used to combine DAS and multi-component geophone data. The challenges of modeling DAS in 3D are also discussed including recording the proper component of strain along the fiber. The resulting migrated images did not clarify if DAS added any qualitative value to multi-component geophone images as the migrated images were too low of a frequency to analyze due to the limitation of the velocity model. A quantitative analysis of the combined image is utilized. This chapter concluded that adding DAS data only helped to reduce the number of false positives by a very small fraction. This experiment is inconclusive in regards to identifying if DAS can add value to sparsely sampled geophone data, so another experiment was performed with 2D surface DAS fiber lines. The new proposed experiment with long-offset, 2D surface fiber lines concluded that using the new geometry was better suited for surface DAS acquisitions. The next proposed step is to test this hypothesis with field data.
